\documentclass[12pt, twosided]{article}
\input{sdpreamble}
\graphicspath{{./img/}}

\begin{document}
\noindent \textbf{Commutative Algebra} \hfill \textbf{} \\
\textbf{Scribed by: Kyle Dituro} \hfill \textbf{Updated \today}\hrule
\vspace{.2in}

\section{A Review of Rings}
We do a quick, informal review of things that we should know already.

\begin{df}
  A \textbf{Ring} \(R\) is (put simply) a set with two binary operations.
\end{df}

\begin{df}
  An \textbf{ideal} is a subset of a ring which is closed under addition and has the absorption property of multiplication, namely \[RI \subseteq I\]
\end{df}

\begin{df}
  A \textbf{quotient ring by an ideal \(I\)} denoted \(R/I\) is the ring whose elements are the cosets \(x + I\), often denoted by \(\overline{x}\). It is a frequent exercise to check that this notation is non-ambiguous.
\end{df}

\begin{prop}\label{Ex1}
  There is a 1-1 correspondence between ideals in \(R/I\) and ideals in \(R\) containing \(I\).
\end{prop}
\begin{proof}
  {\color{red} \textit{(Exercise) Hint: think about the projection map.}}
\end{proof}

A number particular types of ideals will be of use to us over the course of this class, namely:

\begin{enumerate}
\item \emph{Principle}: Generated by a single element
\item \emph{Maximal}: Nontrivial ideal which isn't in any others
\item \emph{Prime Ideal:} An ideal \(P\) such that \(x \not\in P\) and \(y \not\in P\) then \(xy \not\in P\). Moreover if \(xy \in P\), then \(x\in P\) or \(y \in P\). E.g. take any ideal in \(\Z\) generated by a prime number.
\end{enumerate}

Notice that the structure of quotient rings by ideals classifies nicely:

\begin{thm}[Quotient Ring Classification]\label{Ex2} \hspace{0ex}
  \begin{enumerate}
  \item \(R/M\) is a field \(\longleftrightarrow\) \(M\) is maximal
  \item \(R/P\) is an Integral domain \(\longleftrightarrow\) \(P\) is prime.
  \end{enumerate}
\end{thm}
\begin{proof} Solution given by Saskia:
  \begin{enumerate}
  \item [(1.)]\hspace{0ex}
    \begin{enumerate}
    \item [(\(\Rightarrow\))] \(R/M\) is a field, so it therefore has no nontrivial ideals. Then also, by the correspondence theorem \(R\) has no ideals \(I\) with \(M \subset I \subset R\).
    \item [(\(\Leftarrow\))] Works the same way as forward direction.\partdone
    \end{enumerate}
  \item [(2.)]\hspace{0ex}
    \begin{enumerate}
    \item [\(\Rightarrow\)]. We have that \([x],[y] \in R/M\), so then \(x_1 \in [x], y_1 \in [y]\), but then \([x_1y_1] = [xy] = 0\) iff \([x]\) or \([y] = 0\). Then, by definition of an ideal, \(x_1y_1 \in M \Leftrightarrow x_1\) or \(y_1 \in M\).
    \item [(\(\Leftarrow\))] Since \(M\) is a prime ideal, \(xy \in M\) means that one of \(x\) or \(y\) is in \(M\), then \([xy] = [x][y] = 0 \Leftrightarrow [x]=0\) or \([y] = 0\)
    \end{enumerate}
  \end{enumerate}
\end{proof}

\begin{proof}
  Alternative proof of the field condition given by Ishaan.

  \begin{enumerate}
  \item [(\(\Rightarrow\))] Let \(I\) be an ideal such that \(M \subseteq I\). Then let \(a \in I\) be given such that \(a \not\in M\). Then \(a + M \neq 0 + M\), and also \(\exists b + M\) such that \((a + M)(b + M) = 1 + M\). Then \(ab + M  = 1+M\), which indicates that \(ab - 1 \in M \subsetneq I\). Thus \(ab \in I, 1 \in I\), so \(1 \in I = R\).
  \item [(\(\Leftarrow\))] Suppose that \(M\) is maximal, then let \(a + M \in R/M\) such such that \(a + M \neq 0 + M\). Now notice that if we attempt to extend the ideal \(M\) via \(a\), we will get the whole ring. So then since \(1 \in R\), \(\exists r \in R\), \(\exists m \in M\) such that \(1 = ra + m\), so \(1 + M = ra + M\), and \((r + M)(a + M)\).
  \end{enumerate}
\end{proof}
\begin{note}
  Every maximal ideal is prime.
\end{note}

\begin{prop}
  Every ring has a maximal ideal.
\end{prop}
\begin{proof}
  {\color{red} \textit{(Exercise)}}
\end{proof}

\begin{cor}
  Every ideal is contained in some maximal ideal
\end{cor}

\begin{cor}
  Every non-unit is contained in a maximal ideal
\end{cor}

\begin{df}
  A \textbf{local ring} is a ring which has a single maximal ideal
\end{df}

\begin{prop}
  Suppose \(m \subseteq R\) such that every \(\alpha \in m\c\) is a unit. Then \(R\) is local with maximal ideal \(m\).
\end{prop}

\begin{exa}
  Let \(C^0(\R) = \opn  \text{continuous functions} j: \R \to \R \cls\), and take the ring identified by the collection of germs \([f]_0\) which are centered at \(0\).

  Notice that the collection of functions vanishing at \(0\) is a maximal ideal.
\end{exa}
\begin{proof}
  Define the evaluation map at zero \(
  \begin{matrix}
    \mathrm{ev}_0: C^0(R)_0 \to \R \\
    \mathrm{ev}_0(f) \mapsto f(0)
  \end{matrix}.
  \). Notice that for any continuous function \(f \neq 0\) at \(0\)., the \(\frac{1}{f}\) is also continuous at \(0\), and \(f * \frac{1}{f} = 1\).
\end{proof}

\subsection{Exercises}
\begin{enumerate}
\item Prove (\ref{Ex1})
\item Prove (\ref{Ex2})
\item Let \(K\) be a field show that \((f(x)) = (g(x))\) iff \(f\) and \(g\) differ by a constant for \(f(x), g(x) \in K[x]\),
  \begin{proof} This proof was given by Grace.
    \begin{enumerate} 
    \item [(\(\Rightarrow\))]Suppose that \((f(x)) = (g(x))\), \(f(x), g(x) \in K[x]\), then there exists some \(k(x), k\prime(x) \in K[x]\) such that \(f(x) = k(x)g(x)\) and \(g(x) = k\prime(x)f(x)\). Then \(\deg(f(x)) = \deg(k) + \deg(g) = \deg k + \deg k\prime + \deg f\), and so the degree of the \(k\)s must be zero.
    \item [(\(\Leftarrow\))] Suppose that \(\exists c, c\prime \in \R\) such that \(f(x) = cg(x)\) and \(g(x) = c\prime f(x)\)
    \end{enumerate}
  \end{proof}
\item Prove that if \(R\) is an integral domain, then \(R[x]\) is too.

\item Define \(\C[[x]] = \opn \sum_{n=0}^\infty a_nx^n \stbar a_n \in \C\cls\). This is the ring of formal power series. Show that \(\C[[x]]\) is a local ring.

  \begin{proof}
    Consider the ideal of non-units \(M = \langle x \rangle\), i.e. the ideal of all formal power series with constant term \(= 0\).

    It is trivially obvious that this is an ideal.

    Now notice that \(\C[[x]]/M \cong \C\) since \(\forall a \in \C[[x]]\),
    \begin{align*}
      a = \sum_{i=0}^\infty a_ix^i = \underbrace{\left(\sum_{i = 1}^\infty a_ix^i\right)}_{\in \langle x \rangle} + a_0,
    \end{align*}
    so in \(\C[[x]]/M\), \(\overline{a}\) is in bijection with \(a_0\) in \(\C\). Then, since \(\C\) is a field, \(M\) must indeed be maximal. Now suppose that there was some other maximal ideal \(M\prime \neq M\). But if this is the case, then there must be some unit \(u \in M\prime\), which means that \(1 = uu\1 \in M\prime\), so \(M\prime = \C[[x]]\). And thus \(M\) is indeed a unique maximal ideal\footnote{There is an assumption here that \(M\) really is the ideal containing all the non units, but this is easy to show}.
  \end{proof}
\end{enumerate}

\section{Ideal Quotient and Radicals}

\begin{df}[Ideal Quotient]
  The \textbf{ideal quotient} is defined as
  \begin{align*}
    (i:j) &= \opn x \in R \stbar xJ \subseteq I \cls \\
          &= \opn x \in R \stbar xj \in I\ \forall j \in J \cls
  \end{align*}
\end{df}
\begin{exa}
  \((6:18) = \R\)\\
  \((18:6) = (3)\)
\end{exa}
\begin{df}
  \(r(I)\), called the radical of an ideal \(I\) is defined as
  \begin{align*}
    \opn x \in R \stbar x^n \in I \text{ for some } n \in \N \cls
  \end{align*}
\end{df}

\begin{exa}
  \(r(8) = (2)\).
\end{exa}

\begin{prop}\label{Prob2.3}
  For \(m = p_1^{r_1}\ldots p_n^{r_n}\), \(r(m) = (p_1 \cdot p_2 \cdot \ldots p_n)\)
\end{prop}
\begin{proof}
  {\color{red} \textit{Exercise.}}
\end{proof}
\subsection{Algebraic Geometry}

Let \(K\) be an algibraically closed field, and let \(R = k[x_1, \ldots, x_n]\).

\begin{df}
  The \textbf{zero locus} of a polynomial \(f \in R\) is defined as
  \begin{align*}
    Z(j) = \opn p \in K^n \stbar f(p) = 0 \cls
  \end{align*}
\end{df}

\begin{exa}
  \(y - x^2 \in \R[x, y]\) will have a zero locus shaped like a parabola in the \(x-y\) plane.
\end{exa}
  \begin{df}
    For \(A \subseteq K^n\), define
    \begin{align*}
      I(A) = \opn f \in R \stbar f(p) = 0 \forall p \in A \cls
    \end{align*}
  \end{df}
  \begin{exa}
    Taking \(A = \opn -2, 2 \cls\), and we get \((x-2) \cap (x + 2) = (x^2 - 4)\).
  \end{exa}

  Ideally\footnote{pun intended} we would like for there to be a 1-1 corrospondance between the zero loci of polynomials and ideals... Is this so?p

  \begin{exa}
    \(Z(x^2) = 0 = Z(0)\).
  \end{exa}

  Well dang. There goes that idea. Here's a better one:

  \begin{thm}[Hilbert's Nullstelensatz (I spelt this wrong)]
    Let \(I \subseteq R\) and suppose that \(f\) vanishes on all of \(Z(I)\), then \(f^r \in I\) for some \(r\).
  \end{thm}

  \subsection{Exercises}
  \begin{enumerate}
  \item For \(\Z\), show:
    \begin{enumerate}
    \item \((m) + (n) = (\gcd(m,n))\)
    \item \((m) \cap (n) = (\mathrm{lcm}(m, n))\)
    \item \((m)(n) = (mn)\)
    \end{enumerate}
  \item For \(\Z\) \((m:n) = \left(\frac{m}{\gcd(m,n)}\right)\)
  \item Prove Proposition \ref{Prob2.3}
  \item Problem 15 in Atiyah.
  \item Problem 17 in Atiyah.
  \item Problem 19 in Atiyah.
  \end{enumerate}
\end{document}
%%% Local Variables:
%%% mode: latex
%%% TeX-master: t
%%% End:
