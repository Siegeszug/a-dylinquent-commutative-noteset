\documentclass[12pt, twosided]{article}
\usepackage[letterpaper,bindingoffset=0in,%
            left=1in,right=1in,top=1in,bottom=1in,%
            footskip=.25in]{geometry}

\usepackage{mathtools}
\usepackage{graphicx}

\usepackage{setspace}
\setstretch{1.1}

\usepackage{amsmath}
\usepackage{amsfonts}
\usepackage{amsthm}
\usepackage{amssymb}
\usepackage{csquotes}
\usepackage{relsize}

\usepackage{tikz}
\usetikzlibrary{cd}
\usetikzlibrary{fit,shapes.geometric}
\tikzset{%  
    mdot/.style={draw, circle, fill=black},
    mset/.style={draw, ellipse, very thick},
}

\usepackage{hhline}
\usepackage{systeme}
\usepackage{mathrsfs}
\usepackage{hyperref}
\usepackage{mathtools}  
\usepackage{silence}
\usepackage{blkarray}
\usepackage{float}
\usepackage{framed}
\usepackage{array}
\usepackage{stmaryrd}
\usepackage{extarrows}
\usepackage{caption}
\captionsetup[figure]{labelfont={bf},name={Fig.},labelsep=period}

\theoremstyle{definition}
\newtheorem{df}{Definition}[section]
\newtheorem{exa}[df]{Example}
\newtheorem{ques}[df]{Question}
\newtheorem{exr}[df]{Exercise}
\newtheorem{prb}[df]{Problem}
\newtheorem*{notn}{Notation}
\newtheorem*{note}{Note}
\theoremstyle{plain}
\newtheorem{thm}[df]{Theorem}
\newtheorem{prop}[df]{Proposition}
\newtheorem{conj}[df]{Conjecture}
\newtheorem{cor}[df]{Corollary}
\newtheorem{lm}[df]{Lemma}
\newtheorem*{fact}{Fact}
\newtheorem*{idea}{Idea}
\newtheorem*{clm}{Claimn}
\newtheorem*{rmk}{Remark}
\usepackage[ruled]{algorithm2e}

\usepackage{ulem}
\makeatletter

\def\lf{\left\lfloor}   
\def\rf{\right\rfloor}
\def\lc{\left\lceil}   
\def\rc{\right\rceil}
\def\st{\text{ s.t. }}
\def\1{^{-1}}
\def\2{^2}
\def\3{^3}
\def\tn{^n}
\def\ind{\mathbf{1}}
\def\R{\mathbb{R}}
\def\Q{\mathbb{Q}}
\def\Z{\mathbb{Z}}
\def\C{\mathbb{C}}
\def\I{\mathbb{I}}
\def\N{\mathbb{N}}
\def\F{\mathbb{F}}
\def\A{\mathbb{A}}
\def\Li{\text{Li}}
\def\th{^\text{th}}
\def\sp{\text{Sp}}
\def\opn{\left\{}
\def\cls{\right\}}
\def\Aut{\text{Aut}}
\def\PG{\text{PG}}
\def\GL{\text{GL}}
\def\PGL{\text{PGL}}
\def\Cov{\text{Cov}}
\def\Pack{\text{Pack}}
\def\PgamL{\text{P}\Gamma\text{L}}
\def\gamL{\Gamma\text{L}}
\def\cl{\text{cl}}
\def\stbar{\ \middle\vert\ }
\def\partdone{\hphantom{1} \hfill \(\triangle\)}
\def\s0{_0}
\def\s1{_1}
\def\s2{_2}
\def\id{\mathrm{id}}
\def\topn{\text{ open}}
\def\Bd{\text{Bd }}
\def\nope{\(\longrightarrow\!\!\longleftarrow\)}
\def\stt{\(^{\text{st}}\ \)}
\def\tht{\(^{\text{th}}\ \)}
\def\ndt{\(^{\text{nd}}\ \)}
\def\t{^{T}}
\def\c{^c}
\renewcommand{\P}{\mathbb{P}}
\newcommand{\leg}[2]{\left( \frac{#1}{#2} \right)}

\renewcommand*\env@matrix[1][*\c@MaxMatrixCols c]{%
   \hskip -\arraycolsep
   \let\@ifnextchar\new@ifnextchar
   \array{#1}}
\makeatother

% These two lines suppress the warning generated 
% by amsmath for overwriting the choose command  
% because it's annoying. This probably has unint-
% ended ramifications somewhere else, but I'm too
% lazy to actually figure that out, so we'll cro-
% ss that bridge when we come to it lol.
\renewcommand{\choose}[2]{\left( {#1 \atop #2} \right)}
\WarningFilter{amsmath}{Foreign command} 

\renewcommand{\mod}[1]{\ (\mathrm{mod}\ #1)}
\renewcommand{\vec}[1]{\mathbf{#1}}

\let\oldprime\prime
\def\prime{^\oldprime}

\usepackage{float}
\restylefloat{figure}

\usepackage{cleveref}
\Crefname{thm}{Theorem}{Theorems}

% Comment commands for co-authors
\newcommand{\kmd}[1]{{\color{purple} #1}}

\newcolumntype{L}{>{$}l<{$}}
% Bib matter
\let\oldepsilon\epsilon
\def\epsilon{\varepsilon}

\let\oldphi\phi
\def\phi{\varphi}

%%% Local Variables:
%%% mode: plain-tex
%%% TeX-master: t
%%% End:

\graphicspath{{./img/}}

\begin{document}
\noindent \textbf{Commutative Algebra} \hfill \textbf{} \\
\textbf{Scribed by: Kyle Dituro} \hfill \textbf{Updated \today}\hrule
\vspace{.2in}

\section{A Review of Rings}
We do a quick, informal review of things that we should know already.

\begin{df}
  A \textbf{Ring} \(R\) is (put simply) a set with two binary operations.
\end{df}

\begin{df}
  An \textbf{ideal} is a subset of a ring which is closed under addition and has the absorption property of multiplication, namely \[RI \subseteq I\]
\end{df}

\begin{df}
  A \textbf{quotient ring by an ideal \(I\)} denoted \(R/I\) is the ring whose elements are the cosets \(x + I\), often denoted by \(\overline{x}\). It is a frequent exercise to check that this notation is non-ambiguous.
\end{df}

\begin{prop}\label{Ex1}
  There is a 1-1 correspondence between ideals in \(R/I\) and ideals in \(R\) containing \(I\).
\end{prop}
\begin{proof}
  {\color{red} \textit{(Exercise) Hint: think about the projection map.}}
\end{proof}

A number particular types of ideals will be of use to us over the course of this class, namely:

\begin{enumerate}
\item \emph{Principle}: Generated by a single element
\item \emph{Maximal}: Nontrivial ideal which isn't in any others
\item \emph{Prime Ideal:} An ideal \(P\) such that \(x \not\in P\) and \(y \not\in P\) then \(xy \not\in P\). Moreover if \(xy \in P\), then \(x\in P\) or \(y \in P\). E.g. take any ideal in \(\Z\) generated by a prime number.
\end{enumerate}

Notice that the structure of quotient rings by ideals classifies nicely:

\begin{thm}[Quotient Ring Classification]\label{Ex2} \hspace{0ex}
  \begin{enumerate}
  \item \(R/M\) is a field \(\longleftrightarrow\) \(M\) is maximal
  \item \(R/P\) is an Integral domain \(\longleftrightarrow\) \(P\) is prime.
  \end{enumerate}
\end{thm}
\begin{proof}
  {\color{red} \textit{(Exercise)}}
\end{proof}

\begin{note}
  Every maximal ideal is prime.
\end{note}

\begin{prop}
  Every ring has a maximal ideal.
\end{prop}
\begin{proof}
  {\color{red} \textit{(Exercise)}}
\end{proof}

\begin{cor}
  Every ideal is contained in some maximal ideal
\end{cor}

\begin{cor}
  Every non-unit is contained in a maximal ideal
\end{cor}

\begin{df}
  A \textbf{local ring} is a ring which has a single maximal ideal
\end{df}

\begin{prop}
  Suppose \(m \subseteq R\) such that every \(\alpha \in m\c\) is a unit. Then \(R\) is local with maximal ideal \(m\).
\end{prop}

\begin{exa}
  Let \(C^0(\R) = \opn  \text{continuous functions} j: \R \to \R \cls\), and take the ring identified by the collection of germs \([f]_0\) which are centered at \(0\).

  Notice that the collection of functions vanishing at \(0\) is a maximal ideal.
\end{exa}
\begin{proof}
  Define the evaluation map at zero \(
  \begin{matrix}
    \mathrm{ev}_0: C^0(R)_0 \to \R \\
    \mathrm{ev}_0(f) \mapsto f(0)
  \end{matrix}.
  \). Notice that for any continuous function \(f \neq 0\) at \(0\)., the \(\frac{1}{f}\) is also continuous at \(0\), and \(f * \frac{1}{f} = 1\).
\end{proof}

\subsection{Exercises}
\begin{enumerate}
\item Prove (\ref{Ex1})
\item Prove (\ref{Ex2})
\item Let \(K\) be a field show that \((f(x)) = (g(x))\) iff \(f\) and \(g\) differ by a constant for \(f(x), g(x) \in K[x]\), 
\item Prove that if \(R\) is an integral domain, then \(R[x]\) is too.
\item Define \(\C[[x]] = \opn \sum_{n=0}^\infty a_nx^n \stbar a_n \in \C\cls\). This is the ring of formal power series. Show that \(\C[[x]]\) is a local ring.
\end{enumerate}
\end{document}
%%% Local Variables:
%%% mode: latex
%%% TeX-master: t
%%% End:
